%% For double-blind review submission, w/o CCS and ACM Reference (max submission space)
\documentclass[acmsmall,review,anonymous]{acmart}\settopmatter{printfolios=true,printccs=false,printacmref=false}
%% For double-blind review submission, w/ CCS and ACM Reference
% \documentclass[acmsmall,review,anonymous]{acmart}\settopmatter{printfolios=true}
%% For single-blind review submission, w/o CCS and ACM Reference (max submission space)
% \documentclass[acmsmall,review]{acmart}\settopmatter{printfolios=true,printccs=false,printacmref=false}
%% For single-blind review submission, w/ CCS and ACM Reference
% \documentclass[acmsmall,review]{acmart}\settopmatter{printfolios=true}
%% For final camera-ready submission, w/ required CCS and ACM Reference
% \documentclass[acmsmall]{acmart}\settopmatter{}


%% Journal information
%% Supplied to authors by publisher for camera-ready submission;
%% use defaults for review submission.
\acmJournal{PACMPL}
\acmVolume{1}
\acmNumber{CONF} % CONF = POPL or ICFP or OOPSLA
\acmArticle{1}
\acmYear{2018}
\acmMonth{1}
\acmDOI{} % \acmDOI{10.1145/nnnnnnn.nnnnnnn}
\startPage{1}

%% Copyright information
%% Supplied to authors (based on authors' rights management selection;
%% see authors.acm.org) by publisher for camera-ready submission;
%% use 'none' for review submission.
\setcopyright{none}
% \setcopyright{acmcopyright}
% \setcopyright{acmlicensed}
% \setcopyright{rightsretained}
% \copyrightyear{2018}           %% If different from \acmYear

%% Bibliography style
\bibliographystyle{ACM-Reference-Format}
%% Citation style
%% Note: author/year citations are required for papers published as an
%% issue of PACMPL.
\citestyle{acmauthoryear}   %% For author/year citations


%%%%%%%%%%%%%%%%%%%%%%%%%%%%%%%%%%%%%%%%%%%%%%%%%%%%%%%%%%%%%%%%%%%%%% 
%% Note: Authors migrating a paper from PACMPL format to traditional
%% SIGPLAN proceedings format must update the '\documentclass' and
%% topmatter commands above; see 'acmart-sigplanproc-template.tex'.
%%%%%%%%%%%%%%%%%%%%%%%%%%%%%%%%%%%%%%%%%%%%%%%%%%%%%%%%%%%%%%%%%%%%%% 


%% Some recommended packages.
\usepackage{booktabs}   %% For formal tables:
%% http://ctan.org/pkg/booktabs
\usepackage{subcaption} %% For complex figures with subfigures/subcaptions
%% http://ctan.org/pkg/subcaption


\begin{document}

%% Title information
\title{SpecHunter: Interactively Inferring Taint Specifications Using Bayes Net}        
%% when present, will be used in
%% header instead of Full Title.
\titlenote{with title note}             %% \titlenote is optional;
%% can be repeated if necessary;
%% contents suppressed with 'anonymous'
\subtitle{Subtitle}                     %% \subtitle is optional
\subtitlenote{with subtitle note}       %% \subtitlenote is optional;
%% can be repeated if necessary;
%% contents suppressed with 'anonymous'


%% Author information
%% Contents and number of authors suppressed with 'anonymous'.
%% Each author should be introduced by \author, followed by
%% \authornote (optional), \orcid (optional), \affiliation, and
%% \email.
%% An author may have multiple affiliations and/or emails; repeat the
%% appropriate command.
%% Many elements are not rendered, but should be provided for metadata
%% extraction tools.

%% Author with single affiliation.
\author{First1 Last1}
\authornote{with author1 note}          %% \authornote is optional;
%% can be repeated if necessary
\orcid{nnnn-nnnn-nnnn-nnnn}             %% \orcid is optional
\affiliation{
  \position{Position1}
  \department{Department1}              %% \department is recommended
  \institution{Institution1}            %% \institution is required
  \streetaddress{Street1 Address1}
  \city{City1}
  \state{State1}
  \postcode{Post-Code1}
  \country{Country1}                    %% \country is recommended
}
\email{first1.last1@inst1.edu}          %% \email is recommended

%% Author with two affiliations and emails.
% \author{First2 Last2}
% \authornote{with author2 note}          %% \authornote is optional;
%                                         %% can be repeated if necessary
% \orcid{nnnn-nnnn-nnnn-nnnn}             %% \orcid is optional
% \affiliation{
% \position{Position2a}
% \department{Department2a}             %% \department is recommended
% \institution{Institution2a}           %% \institution is required
% \streetaddress{Street2a Address2a}
% \city{City2a}
% \state{State2a}
% \postcode{Post-Code2a}
% \country{Country2a}                   %% \country is recommended
% }
%   \email{first2.last2@inst2a.com}         %% \email is recommended
%   \affiliation{
%   \position{Position2b}
%   \department{Department2b}             %% \department is recommended
%   \institution{Institution2b}           %% \institution is required
%   \streetaddress{Street3b Address2b}
%   \city{City2b}
%   \state{State2b}
%   \postcode{Post-Code2b}
%   \country{Country2b}                   %% \country is recommended
% }
%   \email{first2.last2@inst2b.org}         %% \email is recommended


%%   Abstract
%%   Note: \begin{abstract}...\end{abstract} environment must come
%%   before \maketitle command
\begin{abstract}
  % Taint analysis becomes a must for Java projects where foreign codes such as
  % external frameworks are frequently used in building them.
  % % Taint specification 추론이 왜 필요한지
  % For accurate taint analysis, the analyzer must first be told which methods are
  % sources, sinks, sanitizers, or none of them. However, Manually determining such
  % labels for every method used or defined in a project is a tedious task that
  % becomes nearly impossible if the codebase becomes large.
  % % 그러므로 우리의 아이디어를 제시한다
  % To alleviate such burden, we present Spechunter which, given a small
  % portion of the entire methods which are not deadcodes, finds out taint specifications
  % of the rest of the methods in a Java project built with frameworks.
  % % 우리 아이디어의 특징
  % Spechunter's uniqueness lies in its efficiency, achieved by only
  % drawing minimum manpower necessary for doing its job. This is possible due to its
  % interactive nature: it ends an ongoing interaction as soon as it determines that enough
  % evidence is obtained from the oracle. To achieve this, Spechunter relies on
  % building Bayesian networks and performing marginal inference on it given accumulating
  % evidence. Our experiments show that ...

  % 위의 버전은 영양가도 없으면서 너무 장황하다. Lets give it a full rewrite!

  SpecHunter is a tool for inferring taint specifications, aiming to serve any
  Java taint analyzers, enabling them to be more precise in finding out data
  flows from a source method to a sink method, without passing through a
  sanitizer. By asking questions to an external oracle on selected APIs used in
  a given Java project, it can acheive the following two goals. First,
  SpecHunter is language-agnostic, since it only focuses on the Java project at
  hand, and ignores whether or not the used APIs are implemented in languages
  other than Java. Therefore, it can be used for any Java projects, regardless
  of what libraries or frameworks they use. Second, SpecHunter is
  efficient, because the APIs necessary for efficient propagation is automatically selected
  by SpecHunter itself. This allows the user to skip all trial-and-error in
  configurating the training set for a learning model, as required by previous
  machine learning approaches. Results of our experiments show that SpecHunter
  actually delivers the aforementioned promises, by running on projects with
  Scala and C++ libraries, with only \todo questions asked to the user.  
\end{abstract}


%% 2012 ACM Computing Classification System (CSS) concepts
%% Generate at 'http://dl.acm.org/ccs/ccs.cfm'.
\begin{CCSXML}
  <ccs2012>
  <concept>
  <concept_id>10011007.10011006.10011008</concept_id>
  <concept_desc>Software and its engineering~General programming languages</concept_desc>
  <concept_significance>500</concept_significance>
  </concept>
  <concept>
  <concept_id>10003456.10003457.10003521.10003525</concept_id>
  <concept_desc>Social and professional topics~History of programming languages</concept_desc>
  <concept_significance>300</concept_significance>
  </concept>
  </ccs2012>
\end{CCSXML}

\ccsdesc[500]{Software and its engineering~General programming languages}
\ccsdesc[300]{Social and professional topics~History of programming languages}
%% End of generated code


%% Keywords
%% comma separated list
\keywords{keyword1, keyword2, keyword3}  %% \keywords are mandatory in final camera-ready submission


%% \maketitle
%% Note: \maketitle command must come after title commands, author
%% commands, abstract environment, Computing Classification System
%% environment and commands, and keywords command.
\maketitle

\section{Introduction}
% A large portion of Java projects are built using frameworks such as Spring,
% Hibernate, or Android. These frameworks are foreign codes, so importing their
% APIs may introduce new security vulnerabilities. Finding out such security holes
% requires running taint analysis, which looks for a data flow starting from a
% source method to a sink method without passing through a sanitizer method.
% However, taint analyzers should be correctly told which method is a source,
% sink, sanitizer, or none of them at the first place, otherwise it may result in
% false negatives or false positives. However, manually labelling all methods used
% or defined in a project with the above four labels is a laborous task to be
% manually done, and becomes nearly impossible to do so if the codebase gets
% larger. Therefore, there have been previous works to lessen this burden.

% 2021-02-15 16:44:59에 다시 쓴 버전:

Taint analysis, whether static or dynamic, aims to find security vulnerabilities lurking
in the given program. These vulnerabilities include SQL injection, and cross-site
scripting, and abused printf-arguments, and they are typically formulated as a
data flow starting from a 



%% Acknowledgments
\begin{acks}                            %% acks environment is optional
  %% contents suppressed with 'anonymous'
  %% Commands \grantsponsor{<sponsorID>}{<name>}{<url>} and
  %% \grantnum[<url>]{<sponsorID>}{<number>} should be used to
  %% acknowledge financial support and will be used by metadata
  %% extraction tools.
  This material is based upon work supported by the
  \grantsponsor{GS100000001}{National Science
    Foundation}{http://dx.doi.org/10.13039/100000001} under Grant
  No.~\grantnum{GS100000001}{nnnnnnn} and Grant
  No.~\grantnum{GS100000001}{mmmmmmm}.  Any opinions, findings, and
  conclusions or recommendations expressed in this material are those
  of the author and do not necessarily reflect the views of the
  National Science Foundation.
\end{acks}


%% Bibliography
% \bibliography{bibfile}


%% Appendix
\appendix
\section{Appendix}

Text of appendix \ldots

\end{document}
