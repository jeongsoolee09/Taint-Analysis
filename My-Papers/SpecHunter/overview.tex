\section{Overview}

% SAVER보다는 MemFix를 참고하는 게 더 나을 것
% Overview에 약 2 페이지가 사용되었다.
% - 약 40%는 motivating example로 채웠고,
% - 나머지 60%는 concrete한 동작 방식으로 채웠다.

\subsection{A Simple Example}

\lstinputlisting[language=Java]{./Codes/OverviewExample.java}

We illustrate our approach with a simple example. Figure \emph{xx} contains a
Spring program that interacts with a DBMS via APIs provided by Spring's
\texttt{JdbcTemplate}. (We assume that necessary modules are imported
beforehand.) Although the program is short, There are two kinds of
data flows present in this program. First is one that starts from
the standart input and comes into the program via \texttt{Scanner.nextLine()},
flows into \texttt{RelationalData.run()}, which passes it to
\texttt{JdbcTemplate.batchUpdate()} for database insertion. The another comes from
the DBMS via \texttt{JdbcTemplate.queryForMap()} for database querying, and then
flows to the \texttt{Logger.info()} function, which emits this information
outside the program, to the standard output.

These data flows pose two security risks: one is an SQL injection (CWE-89),
and another is insertion of sensitive information into log file (CWE-532).
Therefore, for a taint analyzer to detect these data flows, one must mark
\texttt{Scanner.nextLine()} and \texttt{JdbcTemplate.queryForMap()} as sources,
\texttt{JdbcTemplate.batchUpdate()} and \texttt{Logger.info()} as their respective
sinks.

There are 6 APIs in this program, and we would like to label the APIs with 
as least number of hints given from the oracle as possible:

\begin{itemize}
\item \texttt{RelationalData.run()} is a source,
\item \texttt{JdbcTemplate.batchUpdate()} is a sink,
\item \texttt{JdbcTemplate.queryForMap()} is a source,
\item \texttt{Logger.info()} is a sink, and
\item \texttt{Map.get()} and \texttt{SpringApplication.run()} are none of the three.
\end{itemize}

\subsection{How SpecHunter works}

