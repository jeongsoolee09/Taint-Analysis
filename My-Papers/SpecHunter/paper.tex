%% For double-blind review submission, w/o CCS and ACM Reference (max submission space)
\documentclass[sigconf,review,anonymous]{acmart}\settopmatter{printfolios=true,printccs=false,printacmref=false}
%% For double-blind review submission, w/ CCS and ACM Reference
% \documentclass[acmsmall,review,anonymous]{acmart}\settopmatter{printfolios=true}
%% For single-blind review submission, w/o CCS and ACM Reference (max submission space)
% \documentclass[acmsmall,review]{acmart}\settopmatter{printfolios=true,printccs=false,printacmref=false}
%% For single-blind review submission, w/ CCS and ACM Reference
% \documentclass[acmsmall,review]{acmart}\settopmatter{printfolios=true}
%% For final camera-ready submission, w/ required CCS and ACM Reference
% \documentclass[acmsmall]{acmart}\settopmatter{}


%% Journal information
%% Supplied to authors by publisher for camera-ready submission;
%% use defaults for review submission.
\acmJournal{PACMPL}
\acmVolume{1}
\acmNumber{CONF} % CONF = POPL or ICFP or OOPSLA
\acmArticle{1}
\acmYear{2018}
\acmMonth{1}
\acmDOI{} % \acmDOI{10.1145/nnnnnnn.nnnnnnn}
\startPage{1}

%% Copyright information
%% Supplied to authors (based on authors' rights management selection;
%% see authors.acm.org) by publisher for camera-ready submission;
%% use 'none' for review submission.
\setcopyright{none}
% \setcopyright{acmcopyright}
% \setcopyright{acmlicensed}
% \setcopyright{rightsretained}
% \copyrightyear{2018}           %% If different from \acmYear

%% Bibliography style
\bibliographystyle{ACM-Reference-Format}
%% Citation style
%% Note: author/year citations are required for papers published as an
%% issue of PACMPL.
\citestyle{acmauthoryear}   %% For author/year citations


%%%%%%%%%%%%%%%%%%%%%%%%%%%%%%%%%%%%%%%%%%%%%%%%%%%%%%%%%%%%%%%%%%%%%% 
%% Note: Authors migrating a paper from PACMPL format to traditional
%% SIGPLAN proceedings format must update the '\documentclass' and
%% topmatter commands above; see 'acmart-sigplanproc-template.tex'.
%%%%%%%%%%%%%%%%%%%%%%%%%%%%%%%%%%%%%%%%%%%%%%%%%%%%%%%%%%%%%%%%%%%%%% 


%% Some recommended packages.
\usepackage{booktabs}   %% For formal tables:
%% http://ctan.org/pkg/booktabs
\usepackage{subcaption} %% For complex figures with subfigures/subcaptions
%% http://ctan.org/pkg/subcaption

\usepackage{listings}

\lstset{
  language=Java,
  numbers=left,
  stepnumber=1,
  firstnumber=1,
  numberfirstline=true
}

\begin{document}

%% Title information
\title{SpecHunter: Interactively Inferring Taint Specifications}        
%% when present, will be used in
%% header instead of Full Title.
\titlenote{with title note}             %% \titlenote is optional;
%% can be repeated if necessary;
%% contents suppressed with 'anonymous'
\subtitle{Subtitle}                     %% \subtitle is optional
\subtitlenote{with subtitle note}       %% \subtitlenote is optional;
%% can be repeated if necessary;
%% contents suppressed with 'anonymous'


%% Author information
%% Contents and number of authors suppressed with 'anonymous'.
%% Each author should be introduced by \author, followed by
%% \authornote (optional), \orcid (optional), \affiliation, and
%% \email.
%% An author may have multiple affiliations and/or emails; repeat the
%% appropriate command.
%% Many elements are not rendered, but should be provided for metadata
%% extraction tools.

%% Author with single affiliation.
\author{First1 Last1}
\authornote{with author1 note}          %% \authornote is optional;
%% can be repeated if necessary
\orcid{nnnn-nnnn-nnnn-nnnn}             %% \orcid is optional
\affiliation{
  \position{Position1}
  \department{Department1}              %% \department is recommended
  \institution{Institution1}            %% \institution is required
  \streetaddress{Street1 Address1}
  \city{City1}
  \state{State1}
  \postcode{Post-Code1}
  \country{Country1}                    %% \country is recommended
}
\email{first1.last1@inst1.edu}          %% \email is recommended

%% Author with two affiliations and emails.
% \author{First2 Last2}
% \authornote{with author2 note}          %% \authornote is optional;
%                                         %% can be repeated if necessary
% \orcid{nnnn-nnnn-nnnn-nnnn}             %% \orcid is optional
% \affiliation{
% \position{Position2a}
% \department{Department2a}             %% \department is recommended
% \institution{Institution2a}           %% \institution is required
% \streetaddress{Street2a Address2a}
% \city{City2a}
% \state{State2a}
% \postcode{Post-Code2a}
% \country{Country2a}                   %% \country is recommended
% }
%   \email{first2.last2@inst2a.com}         %% \email is recommended
%   \affiliation{
%   \position{Position2b}
%   \department{Department2b}             %% \department is recommended
%   \institution{Institution2b}           %% \institution is required
%   \streetaddress{Street3b Address2b}
%   \city{City2b}
%   \state{State2b}
%   \postcode{Post-Code2b}
%   \country{Country2b}                   %% \country is recommended
% }
%   \email{first2.last2@inst2b.org}         %% \email is recommended


%%   Abstract
%%   Note: \begin{abstract}...\end{abstract} environment must come
%%   before \maketitle command
\begin{abstract}
  SpecHunter is a tool for inferring taint specifications, aiming to aid anyone
  trying to use a Java taint analyzer, but being overwhelmed by the number of APIs that
  should be marked of their specifications to run it. SpecHunter aims to lessen this burden
  by constructing a Bayesian network and performing live interaction with the user, in which
  it systematically picks a method and asks the user of its specification. By taking an
  interactive approach, SpecHunter can run on Java applications that use libraries that are
  implemented in languages other than Java. Also, SpecHunter skips all trial-and-errors
  in configurating the data-set required in using any of the traditional machine-learning approaches.
  Our experiments show that SpecHunter actually delivers those promises.
\end{abstract}


%% 2012 ACM Computing Classification System (CSS) concepts
%% Generate at 'http://dl.acm.org/ccs/ccs.cfm'.
\begin{CCSXML}
  <ccs2012>
  <concept>
  <concept_id>10011007.10011006.10011008</concept_id>
  <concept_desc>Software and its engineering~General programming languages</concept_desc>
  <concept_significance>500</concept_significance>
  </concept>
  <concept>
  <concept_id>10003456.10003457.10003521.10003525</concept_id>
  <concept_desc>Social and professional topics~History of programming languages</concept_desc>
  <concept_significance>300</concept_significance>
  </concept>
  </ccs2012>
\end{CCSXML}

\ccsdesc[500]{Software and its engineering~General programming languages}
\ccsdesc[300]{Social and professional topics~History of programming languages}
%% End of generated code


%% Keywords
%% comma separated list
\keywords{keyword1, keyword2, keyword3}  %% \keywords are mandatory in final camera-ready submission


%% \maketitle
%% Note: \maketitle command must come after title commands, author
%% commands, abstract environment, Computing Classification System
%% environment and commands, and keywords command.
\maketitle

\section{Introduction}
Taint analysis, whether static or dynamic, aims to find security vulnerabilities lurking
in the given program. These vulnerabilities include SQL injection, cross-site
scripting, and abused printf-arguments. They are typically formulated as a
data flow starting from a `source' method to a `sink' method, without the
flowing data being `sanitized' by a relevant method. For such search to be
precise enough, the analyzer must first be told which methods are sources,
sinks, sanitizers, or none of them. However, going through the codebase
searching for imported APIs and labelling every one of them by hand quickly becomes tedious
thing to do as the codebase gets larger.

% TODO BibTeX을 쓸 줄 모르므로 일단 툴 이름으로 Reference로의 링크를 대신함
% Boldface들을 BibTeX으로 바꿀 것.
To aid such manpower, previous studies proposed methods such as supervised
learning (\textbf{SuSI}, \textbf{SWAN}), and semi-supervised learning (\textbf{Seldon}, \textbf{Merlin}).
However, works belonging to each camp suffers from their own limitations.

\paragraph{Shortcomings of Supervised Learning Camp.}

\textbf{SuSI} and \textbf{SWAN} both work in a similar way: they first train a learning model
such as a support vector machine with the actual library codes implementing
APIs. Here, the function bodies are interprocedurally analyzed with a Java
static analyzer to get the vectors of feature values. Then, they take a whole
library codebase into the pre-trained learning
model, and get the mapping from all APIs in the library to their respective
specifications. While they work nicely against libraries written in pure Java
code such as Spring, they cannot work if a given application imports libraries
written in other languages other than Java. This is because in order to extract
the feature values they would have to be equipped with all the possible
languages the libraries would be in, and this is practically difficult.

% TODO: add an additional shortcoming: features should also be updated if a new
% library is provided

\paragraph{Shortcomings of Semi-Supervised Learning Camp.}

As a workaround to the above limitation, one can try a semi-supervised learning
approach. The state-of-the-art belonging to this camp is \textbf{Seldon}.
\textbf{Seldon} only takes the application code into consideration, thus it does
not care which language the library is written in. So, for example, Seldon can
figure out the label of an API function in a Python program, which is
actually implemented in C++. However, Seldon suffers from a common problem of
all machine-learning solutions, that it is largely dependent on quality dataset.
In fact, the authors reports that \textbf{Seldon}'s precision dropped by 14
percent-points by just providing half of the original specifications. Moreover,
this problem is shared with supervised learning solutions, those are also
data-driven solutions.
% Seldon 논문의 부분을 인용하고 싶은데... 형식을 모르겠네
% 해당 부분:
% Q6: Impact of Seed Specification. We also evaluated Sel-
% don’s precision for half the seed specification (considering
% only odd line numbers in App. B). This significantly reduces
% precision, by 14 percentage-points. Thus, we believe our
% seed specification strikes a good balance between manual
% effort and precision. We note that in the extreme case of an
% empty seed specification, Seldon will predict 0 specifications,
% because picking 0 for all variables is a trivial solution to its
% constraint system.

\paragraph{Our Tool: Real-Time Interactive Inference.}

Our tool, SpecHunter, works in a different manner than data-driven
approaches. Instead of training the model with some predetermined pairs of
methods and their labels to predict other methods' labels, we analyze the given
application to construct probabilistic graphical models of a certain type,
called Bayesian networks. They are special in that they provide a model of
conditional-probabilistic reasoning, which infers the marginal probabilities of
each random variables when some of them are instantiated to a certain value by
an external source. Therefore, we first determine a structure of a Bayesian network
by running a static analysis on the given application code, and make the system ask
the user, the external entity, for evidence. Then, SpecHunter picks the most
influential node which has the most dependent nodes. This makes the interaction
session both effective and efficient.

The experiments show that SpecHunter effectively infers taint specifications for
methods that are untold of its label, with \textbf{xx}\% precision when told only
\textbf{17.7}\% of all APIs.


\paragraph{Contributions.}

\begin{itemize}
  \item We provide SpecHunter
  \item We provide src/sin/san/non ground truths
  \item We provide a taint analyzer (is it possible?)
  \item We make the tool open source (is it possible?)
\end{itemize}

\section{Overview}

\paragraph{Motivation}

% TODO
Swan sucks: too much variance on its performance depending on dataset quality/quantity

\subsection{Overview of our System}

% 적당한 figure 하나 넣자

Figure 1 shows the overall workflow of SpecHunter. Given a Java application's
source code, SpecHunter labels all API methods provided by the library the application
uses. There are three main issues regarding building an effective interactive system:
namely, constructing the network, effectively propagating the evidence given from the
oracle, and making the system scale to run on large input applications.
% Question: Should we define what "API" means?

\subsection{Network Construction}

\paragraph{Determining Edges}

% Network Construction은 반드시 Scalability and Efficiency와 연관되게 되어 있다.
% 그 점을 유념하면서 쓰자: 적절한 pointer 달아 주기.
Here we describe how we represent the input java project into a large graph form, which will
later be turned into a series of Bayesian networks. How we splitt this large
graph into several Bayesian networks will be discussed in the next section. % TODO 좀 더 자연스러운 포인터??
We hope to create a graph $G=(V, E)$,
where $V$ is the set of all methods used or defined in the given application, and $E$ is a
subset of $V\times V$, whose elements are gathered in three different ways. 

First, there are data-flow-edges. We perform a variant of a data-flow analysis on the
given input application to calculate lifetimes of all access paths found in the
entire codes, starting from being defined and ending by being ``dead'', between
which the access path may be ``redefined''.
% 여기서 간단한 예제 넣어주기

\lstinputlisting[language=Java]{./Codes/SimpleExample.java}

In the above example, a variable \texttt{x} is being defined to hold the value \texttt{1} in the method
\texttt{f}, using another method \texttt{m1}. This value flows to \texttt{g}, where this
value is redefined as \texttt{2} using the method \texttt{m2}. After this redefined
value flows to method \texttt{m3} via \texttt{h}, it is no longer used after the call to
\texttt{println}. Here, our goal is to capture this lifecycle. To compute these
for all access paths in the program, we first run a static analysis designed as follows:

\begin{align*}
  A &\in \mathbb{D} &= \mathbb{C}\rightarrow\mathcal{P}(State)\\
  s &\in State &= Procname\times Var\times Loc\times Alias\\
  p &\in Procname &\subseteq Var\\
  a &\in Alias &\subseteq \mathcal{P}(Var)
\end{align*}

For example, for an assignment statement \texttt{int a = b;} inside a procedure
\texttt{f} at line 3, we obtain the tuple set \texttt{\{(f, a, 3, \{a, b\})\}}.
The fourth component of the tuple means that \texttt{a} and \texttt{b} are
aliases. After we create such tuples for each program point, we finally construct
\emph{propagation chains} by threading the tuples via alias relations. The
relevant generated tuples for our goal is:

\begin{itemize}
\item \texttt{\{(f, x, 14, \{x, y\})\}}
\item \texttt{\{(g, y, 18, \{y, u1\})\}}
\item \texttt{\{(m2, u1, 5, \{u1\})\}}
\item \texttt{\{(m2, u1, 6, \{u1\})\}}
\item \texttt{\{(g, z, 6, \{z, u1, w\})\}}
\item \texttt{\{(g, w, 23, \{w, u2\})\}}
\item \texttt{\{(m3, u2, 9, \{u2, println\_1\})\}}
\end{itemize}

>
where \texttt{println\_1} is a Mangled parameter variable of \texttt{println}.
Now, connecting the tuples with the aliases, we get two chains:
\texttt{x -> y -> u1} and \texttt{u1 -> z -> w -> u2 -> println\_1}. Since \texttt{u1} is
redefined in \texttt{m2}, we glue the two together to get the originally desired
information: \texttt{x -> y -> u1 -> z -> w -> u2 -> println\_1}. This translates
into data-flow edges: \texttt{(f, g)}, \texttt{(f, h)}, and \texttt{(m3, println)}.

% TODO 아 이거 실제로 돌려봐서 확인해야 할 거 같은데.. 실제로 이렇게 나오나...?


% TODO static callee가 무엇인지를 좀 알아와서 디테일을 추가할 것.
Second, there are call-edges. Since we used Facebook's Infer (fbinfer.com/) to
implement our static analysis, we used Infer's facility to compute static
callees of a given method to draw the entire callgraph of a given application
code. For each caller \texttt{f} and its callee \texttt{g}, we add an edge
\texttt{(f, g)}.


Last but not least, there are similarity-edges. The idea is to take any two
methods and score its pairwise similarity, and leave only the pairs with score
above a predefined threshold. We measure them against two large groups of
criteria: the ``syntactic features'' and ``semantic features''. For the
syntactic features, we borrrowed largely from \textbf{Swan}'s features that
check how names of methods or their return types are composed. The below
figure % TODO: Figure 번호 달기...
shows all the features we used.

% TODO 표 삐져나간다..!!!!
\begin{center}
  \begin{tabular} { |c|c|c| }
    \hline
    Syntactic & Semantic\\
    \hline
    \hline
    Both has parameters? & Both has at least one same callee?\\
    Both has return type? & Both are making calls but not passing data?\\
    Both contains same word in method name? & Both are being called but not being passed data?\\
    Both contains same word in class name? & Both are making data flow calls?\\
    Both has return type contained in parameter type? & Both are being passed data and passing data simultaneously?\\
    Both has same return type? & Both are being passed data which gets dead in them?\\
    Both has return type contained in method name? & Both has variable redefinitions?\\
    \hline
  \end{tabular}
\end{center} 

By combining all three kinds of edges, we get a finished graph where every node
is represents a node, and every edge represent how one method relates to another.

\paragraph{Determining CPTs}

After we determine which and which methods should be connected with a directed
edge, we then define a conditional probability table (CPT) for each directed
edge. This step is essential since CPTs are essential components of Bayesian
Networks and we want our big graph to be broken down into small
Bayesian networks. Each node should have its own CPT, and it express how the
node's marginal distribution depends
on those of the parent nodes. We use largely two different CPT schemas:
that of the dataflow-edges and those of the call- and similarity-edges.
Regardless of the schemas, the CPT's size is $4\times 4^n$, where $n$ is the
number of parent nodes of a node. The below figure demonstrates a CPT of a node
with two parent nodes, each shooting a data-flow edge to this node.

% TODO 텍스트로만 써 놓은 걸 실제로 색칠하기.
% VH -> Red, High -> Orange,
% Low -> Green, VL -> Blue
\begin{center}
  \begin{tabular} { |c|c|c|c|c|c|c|c|c|c|c|c|c|c|c|c| }
    \hline
    vl   & vl   & vl   & vl   & vl   & vl   & vl   & vl   & vl   & vl   & vl   & vl   & vl   & vl   & vl   & vl  \\
    high & vl   & high & high & vl   & vl   & vl   & vl   & high & vl   & high & high & high & vl   & high & high\\
    high & vl   & vl   & high & vl   & vl   & vl   & vl   & vl   & vl   & vl   & vl   & high & vl   & vl   & high\\
    high & vl   & high & high & vl   & vl   & vl   & vl   & high & vl   & high & high & high & vl   & high & high\\
    \hline
  \end{tabular}
\end{center}

Let us first start with the data-flow schemas. In the above table, red cells
should contain very high probabilities, orange
cells should contain relatively high probabilities, whereas green cells should
be assigned low probabilities, and blue cells should be of very low probabilities.
The idea is this: by the definitions of ``source'', ``sanitizer'', and ``sink'',
data must flow from a source to a sink through possibly a sanitizer, but it
cannot flow in any other sequence of such methods. (For example, data cannot
possibly flow from a sink through a source to a sanitizer.) Also, we note that
any number of ``none'' methods, which are none of the three, can be interleaved
freely between a source and a sanitizer, and between a sanitizer and a sink.
Therefore, we encode these possibilities into a tabular form containing probabilities
like the above.

For the call-edges and similarity-edges, we use the same schema for both. The
below figure shows how we assign probabilities when a node has two parents, each
shooting call-edges or similarity-edges to the node.

% TODO 텍스트로만 써 놓은 걸 실제로 색칠하기.
% VH -> Red, High -> Orange,
% Low -> Green, VL -> Blue
\begin{center}
  \begin{tabular} { |c|c|c|c|c|c|c|c|c|c|c|c|c|c|c|c| }
    \hline
    high & vl   & vl   & vl   & vl   & vl   & vl   & vl   & vl   & vl   & vl   & vl   & vl  & vl   & vl   & vl  \\
    vl   & vl   & vl   & vl   & vl   & high & vl   & vl   & vl   & vl   & vl   & vl   & vl  & vl   & vl   & vl  \\
    vl   & vl   & vl   & vl   & vl   & vl   & vl   & vl   & vl   & vl   & high & vl   & vl  & vl   & vl   & vl  \\
    vl   & vl   & vl   & vl   & vl   & vl   & vl   & vl   & vl   & vl   & vl   & vl   & vl  & vl   & vl   & high\\
    \hline
  \end{tabular}
\end{center}
% TODO 구현 상으로는 위의 high cell이 low cell로 되어 있는데, 나중에 *꼭*
% 바꾸자. high cell로 두는 게 맞는 것 같다.

% explanation: why do we make similarity edges share this schema with call edges?
Combining these together, we can express the case of a node with two parents,
one shooting a data-flow edge while another shoots a call- or similarity edge.

% TODO 텍스트로만 써 놓은 걸 실제로 색칠하기.
% VH -> Red, High -> Orange,
% Low -> Green, VL -> Blue
\begin{center}
  \begin{tabular} { |c|c|c|c|c|c|c|c|c|c|c|c|c|c|c|c| }
    \hline
    low  & vl   & vl   & vl   & low  & vl   & vl   & vl   & low  & vl   & vl   & vl   & low  & vl   & vl   & vl  \\
    high & low  & high & high & vl   & low  & vl   & vl   & high & low  & cell & high & high & low  & high & high\\
    high & vl   & low  & high & vl   & vl   & low  & vl   & vl   & vl   & low  & vl   & high & vl   & low  & high\\
    high & vl   & high & vh   & vl   & vl   & vl   & low  & high & vl   & high & vh   & high & vl   & high & vh  \\
    \hline
  \end{tabular}
\end{center}
% TODO 여기도 구현 상으로는 위의 high cell이 low cell로 되어 있는데, 나중에 *꼭*
% 바꾸자. high cell로 두는 게 맞는 것 같다.

Since a CPT has a constraint that the sum of probabilities assigned to each cell
of a column is 1, we can come up with the following formula to calculate a
concrete probability value for a cell's relative magnitude of probability.
Suppose we call four cells of a column $cell_1$, $cell_2$, $cell_3$, $cell_4$,
then the probability value of one of the four cells is:

\begin{equation*}
  prob(cell) = \frac{mag(cell)\times coeff(mag(cell))}{\sum_{i=1}^{4}mag(cell)\times coeff(mag(cell_i))}
\end{equation*}

where $mag$ is the score based on relative magnitude: we give $1$ to cells whose
probability should be ``very low'', and $4$ to cells whose probability should be
``very high''. $coeff$ of a relative magnitude is a constant that is multiplied
to the magnitude score. It helps to widen the gap between probabilities assigned
to cells in a column, bearing different relative magnitudes.

\subsection{Graph Manipulation}

% TODO cite networkx 
Here, we explain how we cut the big graph into a handful of small ones, and
further preprocess them before turning each into a Bayesian network. We used
\emph{networkx}'s implementation for each core algorithms: Kernighan-Lin
bisection and cycle detection.

\paragraph{Splitting Graphs}

% TODO: self_question_n_answer.py가 interaction 한번에 얼마나 걸리는지를 측정해
% 그래프로 그리기
After we are done making the initial graph, we chop it down to small graphs of
size under a certain threshold, which will later be converted into Bayesian
networks. This breaking down is necessary since the time consumed during loopy
belief propagation and D-separation algorithms, which will be discussed later
in the next section, gets increasingly inefficient as the graph size increases. Since we are aiming
to reduce the time for a single interaction under \emph{xx??} seconds, we
decided that the optimal threshold for a graph is \emph{180??}. The algorithm
for this task is described in figure \emph{??}. This algorithm relies heavily on
the Kernighan-Lin partitioning algorithm, % TODO: should we cite the algorithm's paper?
which partitions the given graph $G=(V, E)$ into two graphs $G_1=(V_1,E_1)$ and
$G_2=(V_2, E_2)$, such that $V_1$ and $V_2$ are almost equal and the edges
connecting $G_1$ and $G_2$ are minimized. % TODO: at least skim the Kernighan-Lin paper!
% is it really 'minimized'??

\paragraph{Decycling, and Controlling in-edges}

Since Bayesian networks are directed acyclic graphs (DAGs), we need to eliminate
all cycles before we turn the splitted graphs into Bayesian networks. We first
use \emph{networkx}'s cycle
detection algorithm which pinpoints the set of edges forming a cycle by
searching the graph in a depth-manner for this task. We then randomly pick one of
such edges causing a cycle and delete it. Repeatedly applying the two
operations until the algorithm fails to detect a cycle, we transform the
splitted graphs into DAGs.

Next, we control the number of incoming edges of each node. It is already
mentioned that the size of a node's CPT is exponential to the number of its
parent nodes. To our empirical knowledge, if the number of incoming edges (hence
the number of parent nodes) becomes more than 6, the process of converting the
graph into a Bayesian network becomes drastically intractable. Therefore, we
reduce the number of parents of such ``rich nodes'' by making the parents of the
rich node shoot a similarity edge not to the rich node in question, but to some
other node that is not yet rich.

Also, we minimize the number of ``isolated nodes'' that have no incoming edges
nor outgoing edges. Since these nodes' marginal probabilities are independent of
those of any other nodes, their marginals never get updated if any other nodes'
marginals get updated. Thus, they are one of the major factors that make the
inference inefficient. We control the number of isolated nodes of a splitted
graph by sticking the isolated node $n_1$ to a node $n_2$ of other graph if both
were connected in the initial graph either by edge $(n_1, n_2)$ or $(n_2, n_1)$. 

After all these preprocessing, we are finally ready to turn chopped graphs into
small, yet full-blown Bayesian networks. Each node corresponds to a random
variable $\text{The taint specification of method} M$, whose possible values are
$source$, $sink$, $sanitizer$, and $none$. Now, they are waiting for an external
oracle to instantiate its value to one of these four.

\subsection{Information Propagation}

Here, we expound on the core inference algorithm. Inference on the Bayesian
network is divided into two parts: local propagation and global propagation.
Local propagations pass information given by the oracle within a single graph,
whereas global propagations pass the information from one graph or several to
another. The Python library \emph{Pomegranate} is used to convert graphs into
Bayesian networks and perform local propagations.   % TODO: cite pomegranate

\paragraph{Local Propagations}

\subparagraph{Picking the Next Node}

We pick the next node based on an algorithm called D-separation 

\subparagraph{Overview of the Local Propagation Algorithm}

Local propagations rely on loopy belief propagation algorithm (loopy BP): it is an
approximate version of an exact belief propagation algorithm (sum-product
algorithm), defined on Bayesian networks. Loopy BP may not converge,
nevertheless it provides a nice approximation to the original BP algorithm.
% TODO: cite https://www.jmlr.org/papers/volume6/ihler05a/ihler05a.pdf
Since the algorithm may not converge, we set the maximum number of iterations to
100, the default value set by \emph{Pomegranate}.

% TODO: should we explain a little on loopy BP?

% where should we insert the details of using d-separation?

Figure \emph{xx??} shows our local propagation algorithm. D-separation is used
when we pick the next node to ask of its specification. 

% insert algorithm here

\paragraph{Global Propagations}

Global propagations rely on a process which we call ``knowledge transfer''. It
relies on the 

\section{Approach Details}

\subsection{Overview of our System}

% 적당한 figure 하나 넣자

Figure 1 shows the overall workflow of SpecHunter. Given a Java application's
source code, SpecHunter labels all API methods provided by the library the application
uses. There are three main issues regarding building an effective interactive system:
namely, constructing the network, effectively propagating the evidence given from the
oracle, and making the system scale to run on large input applications.
% Question: Should we define what "API" means?

\subsection{Network Construction}

\paragraph{Determining Edges}

% Network Construction은 반드시 Scalability and Efficiency와 연관되게 되어 있다.
% 그 점을 유념하면서 쓰자: 적절한 pointer 달아 주기.
Here we describe how we represent the input java project into a large graph form, which will
later be turned into a series of Bayesian networks. How we splitt this large
graph into several Bayesian networks will be discussed in the next section. % TODO 좀 더 자연스러운 포인터??
We hope to create a graph $G=(V, E)$,
where $V$ is the set of all methods used or defined in the given application, and $E$ is a
subset of $V\times V$, whose elements are gathered in three different ways. 

First, there are data-flow-edges. We perform a variant of a data-flow analysis on the
given input application to calculate lifetimes of all access paths found in the
entire codes, starting from being defined and ending by being ``dead'', between
which the access path may be ``redefined''.
% 여기서 간단한 예제 넣어주기

\lstinputlisting[language=Java]{./Codes/SimpleExample.java}

In the above example, a variable \texttt{x} is being defined to hold the value \texttt{1} in the method
\texttt{f}, using another method \texttt{m1}. This value flows to \texttt{g}, where this
value is redefined as \texttt{2} using the method \texttt{m2}. After this redefined
value flows to method \texttt{m3} via \texttt{h}, it is no longer used after the call to
\texttt{println}. Here, our goal is to capture this lifecycle. To compute these
for all access paths in the program, we first run a static analysis designed as follows:

\begin{align*}
  A &\in \mathbb{D} &= \mathbb{C}\rightarrow\mathcal{P}(State)\\
  s &\in State &= Procname\times Var\times Loc\times Alias\\
  p &\in Procname &\subseteq Var\\
  a &\in Alias &\subseteq \mathcal{P}(Var)
\end{align*}

For example, for an assignment statement \texttt{int a = b;} inside a procedure
\texttt{f} at line 3, we obtain the tuple set \texttt{\{(f, a, 3, \{a, b\})\}}.
The fourth component of the tuple means that \texttt{a} and \texttt{b} are
aliases. After we create such tuples for each program point, we finally construct
\emph{propagation chains} by threading the tuples via alias relations. The
relevant generated tuples for our goal is:

\begin{itemize}
\item \texttt{\{(f, x, 14, \{x, y\})\}}
\item \texttt{\{(g, y, 18, \{y, u1\})\}}
\item \texttt{\{(m2, u1, 5, \{u1\})\}}
\item \texttt{\{(m2, u1, 6, \{u1\})\}}
\item \texttt{\{(g, z, 6, \{z, u1, w\})\}}
\item \texttt{\{(g, w, 23, \{w, u2\})\}}
\item \texttt{\{(m3, u2, 9, \{u2, println\_1\})\}}
\end{itemize}

>
where \texttt{println\_1} is a Mangled parameter variable of \texttt{println}.
Now, connecting the tuples with the aliases, we get two chains:
\texttt{x -> y -> u1} and \texttt{u1 -> z -> w -> u2 -> println\_1}. Since \texttt{u1} is
redefined in \texttt{m2}, we glue the two together to get the originally desired
information: \texttt{x -> y -> u1 -> z -> w -> u2 -> println\_1}. This translates
into data-flow edges: \texttt{(f, g)}, \texttt{(f, h)}, and \texttt{(m3, println)}.

% TODO 아 이거 실제로 돌려봐서 확인해야 할 거 같은데.. 실제로 이렇게 나오나...?


% TODO static callee가 무엇인지를 좀 알아와서 디테일을 추가할 것.
Second, there are call-edges. Since we used Facebook's Infer (fbinfer.com/) to
implement our static analysis, we used Infer's facility to compute static
callees of a given method to draw the entire callgraph of a given application
code. For each caller \texttt{f} and its callee \texttt{g}, we add an edge
\texttt{(f, g)}.


Last but not least, there are similarity-edges. The idea is to take any two
methods and score its pairwise similarity, and leave only the pairs with score
above a predefined threshold. We measure them against two large groups of
criteria: the ``syntactic features'' and ``semantic features''. For the
syntactic features, we borrrowed largely from \textbf{Swan}'s features that
check how names of methods or their return types are composed. The below
figure % TODO: Figure 번호 달기...
shows all the features we used.

% TODO 표 삐져나간다..!!!!
\begin{center}
  \begin{tabular} { |c|c|c| }
    \hline
    Syntactic & Semantic\\
    \hline
    \hline
    Both has parameters? & Both has at least one same callee?\\
    Both has return type? & Both are making calls but not passing data?\\
    Both contains same word in method name? & Both are being called but not being passed data?\\
    Both contains same word in class name? & Both are making data flow calls?\\
    Both has return type contained in parameter type? & Both are being passed data and passing data simultaneously?\\
    Both has same return type? & Both are being passed data which gets dead in them?\\
    Both has return type contained in method name? & Both has variable redefinitions?\\
    \hline
  \end{tabular}
\end{center} 

By combining all three kinds of edges, we get a finished graph where every node
is represents a node, and every edge represent how one method relates to another.

\paragraph{Determining CPTs}

After we determine which and which methods should be connected with a directed
edge, we then define a conditional probability table (CPT) for each directed
edge. This step is essential since CPTs are essential components of Bayesian
Networks and we want our big graph to be broken down into small
Bayesian networks. Each node should have its own CPT, and it express how the
node's marginal distribution depends
on those of the parent nodes. We use largely two different CPT schemas:
that of the dataflow-edges and those of the call- and similarity-edges.
Regardless of the schemas, the CPT's size is $4\times 4^n$, where $n$ is the
number of parent nodes of a node. The below figure demonstrates a CPT of a node
with two parent nodes, each shooting a data-flow edge to this node.

% TODO 텍스트로만 써 놓은 걸 실제로 색칠하기.
% VH -> Red, High -> Orange,
% Low -> Green, VL -> Blue
\begin{center}
  \begin{tabular} { |c|c|c|c|c|c|c|c|c|c|c|c|c|c|c|c| }
    \hline
    vl   & vl   & vl   & vl   & vl   & vl   & vl   & vl   & vl   & vl   & vl   & vl   & vl   & vl   & vl   & vl  \\
    high & vl   & high & high & vl   & vl   & vl   & vl   & high & vl   & high & high & high & vl   & high & high\\
    high & vl   & vl   & high & vl   & vl   & vl   & vl   & vl   & vl   & vl   & vl   & high & vl   & vl   & high\\
    high & vl   & high & high & vl   & vl   & vl   & vl   & high & vl   & high & high & high & vl   & high & high\\
    \hline
  \end{tabular}
\end{center}

Let us first start with the data-flow schemas. In the above table, red cells
should contain very high probabilities, orange
cells should contain relatively high probabilities, whereas green cells should
be assigned low probabilities, and blue cells should be of very low probabilities.
The idea is this: by the definitions of ``source'', ``sanitizer'', and ``sink'',
data must flow from a source to a sink through possibly a sanitizer, but it
cannot flow in any other sequence of such methods. (For example, data cannot
possibly flow from a sink through a source to a sanitizer.) Also, we note that
any number of ``none'' methods, which are none of the three, can be interleaved
freely between a source and a sanitizer, and between a sanitizer and a sink.
Therefore, we encode these possibilities into a tabular form containing probabilities
like the above.

For the call-edges and similarity-edges, we use the same schema for both. The
below figure shows how we assign probabilities when a node has two parents, each
shooting call-edges or similarity-edges to the node.

% TODO 텍스트로만 써 놓은 걸 실제로 색칠하기.
% VH -> Red, High -> Orange,
% Low -> Green, VL -> Blue
\begin{center}
  \begin{tabular} { |c|c|c|c|c|c|c|c|c|c|c|c|c|c|c|c| }
    \hline
    high & vl   & vl   & vl   & vl   & vl   & vl   & vl   & vl   & vl   & vl   & vl   & vl  & vl   & vl   & vl  \\
    vl   & vl   & vl   & vl   & vl   & high & vl   & vl   & vl   & vl   & vl   & vl   & vl  & vl   & vl   & vl  \\
    vl   & vl   & vl   & vl   & vl   & vl   & vl   & vl   & vl   & vl   & high & vl   & vl  & vl   & vl   & vl  \\
    vl   & vl   & vl   & vl   & vl   & vl   & vl   & vl   & vl   & vl   & vl   & vl   & vl  & vl   & vl   & high\\
    \hline
  \end{tabular}
\end{center}
% TODO 구현 상으로는 위의 high cell이 low cell로 되어 있는데, 나중에 *꼭*
% 바꾸자. high cell로 두는 게 맞는 것 같다.

% TODO explanation: why do we make similarity edges share this schema with call
% edges?



Combining these together, we can express the case of a node with two parents,
one shooting a data-flow edge while another shoots a call- or similarity edge.

% TODO 텍스트로만 써 놓은 걸 실제로 색칠하기.
% VH -> Red, High -> Orange,
% Low -> Green, VL -> Blue
\begin{center}
  \begin{tabular} { |c|c|c|c|c|c|c|c|c|c|c|c|c|c|c|c| }
    \hline
    low  & vl   & vl   & vl   & low  & vl   & vl   & vl   & low  & vl   & vl   & vl   & low  & vl   & vl   & vl  \\
    high & low  & high & high & vl   & low  & vl   & vl   & high & low  & cell & high & high & low  & high & high\\
    high & vl   & low  & high & vl   & vl   & low  & vl   & vl   & vl   & low  & vl   & high & vl   & low  & high\\
    high & vl   & high & vh   & vl   & vl   & vl   & low  & high & vl   & high & vh   & high & vl   & high & vh  \\
    \hline
  \end{tabular}
\end{center}
% TODO 여기도 구현 상으로는 위의 high cell이 low cell로 되어 있는데, 나중에 *꼭*
% 바꾸자. high cell로 두는 게 맞는 것 같다.

Since a CPT has a constraint that the sum of probabilities assigned to each cell
of a column is 1, we can come up with the following formula to calculate a
concrete probability value for a cell's relative magnitude of probability.
Suppose we call four cells of a column $cell_1$, $cell_2$, $cell_3$, $cell_4$,
then the probability value of one of the four cells is:

\begin{equation*}
  prob(cell) = \frac{mag(cell)\times coeff(mag(cell))}{\sum_{i=1}^{4}mag(cell)\times coeff(mag(cell_i))}
\end{equation*}

where $mag$ is the score based on relative magnitude: we give $1$ to cells whose
probability should be ``very low'', and $4$ to cells whose probability should be
``very high''. $coeff$ of a relative magnitude is a constant that is multiplied
to the magnitude score. It helps to widen the gap between probabilities assigned
to cells in a column, bearing different relative magnitudes.

\subsection{Graph Manipulation}

% TODO cite networkx 
Here, we explain how we cut the big graph into a handful of small ones, and
further preprocess them before turning each into a Bayesian network. We used
\emph{networkx}'s implementation for each core algorithms: Kernighan-Lin
bisection and cycle detection.

\paragraph{Splitting Graphs}

% TODO: self_question_n_answer.py가 interaction 한번에 얼마나 걸리는지를 측정해
% 그래프로 그리기
After we are done making the initial graph, we chop it down to small graphs of
size under a certain threshold, which will later be converted into Bayesian
networks. This breaking down is necessary since the time consumed during loopy
belief propagation and D-separation algorithms, which will be discussed later
in the next section, gets increasingly inefficient as the graph size increases. Since we are aiming
to reduce the time for a single interaction under \emph{xx??} seconds, we
decided that the optimal threshold for a graph is \emph{180??}. The algorithm
for this task is described in figure \emph{??}. This algorithm relies heavily on
the Kernighan-Lin partitioning algorithm, % TODO: should we cite the algorithm's paper?
which partitions the given graph $G=(V, E)$ into two graphs $G_1=(V_1,E_1)$ and
$G_2=(V_2, E_2)$, such that $V_1$ and $V_2$ are almost equal and the edges
connecting $G_1$ and $G_2$ are minimized. % TODO: at least skim the Kernighan-Lin paper!
% is it really 'minimized'??

\paragraph{Decycling, and Controlling in-edges}

Since Bayesian networks are directed acyclic graphs (DAGs), we need to eliminate
all cycles before we turn the splitted graphs into Bayesian networks. We first
use \emph{networkx}'s cycle
detection algorithm which pinpoints the set of edges forming a cycle by
searching the graph in a depth-manner for this task. We then randomly pick one of
such edges causing a cycle and delete it. Repeatedly applying the two
operations until the algorithm fails to detect a cycle, we transform the
splitted graphs into DAGs.

Next, we control the number of incoming edges of each node. It is already
mentioned that the size of a node's CPT is exponential to the number of its
parent nodes. To our empirical knowledge, if the number of incoming edges (hence
the number of parent nodes) becomes more than 6, the process of converting the
graph into a Bayesian network becomes drastically intractable. Therefore, we
reduce the number of parents of such ``rich nodes'' by making the parents of the
rich node shoot a similarity edge not to the rich node in question, but to some
other node that is not yet rich.

Also, we minimize the number of ``isolated nodes'' that have no incoming edges
nor outgoing edges. Since these nodes' marginal probabilities are independent of
those of any other nodes, their marginals never get updated if any other nodes'
marginals get updated. Thus, they are one of the major factors that make the
inference inefficient. We control the number of isolated nodes of a splitted
graph by sticking the isolated node $n_1$ to a node $n_2$ of other graph if both
were connected in the initial graph either by edge $(n_1, n_2)$ or $(n_2, n_1)$. 

After all these preprocessing, we are finally ready to turn chopped graphs into
small, yet full-blown Bayesian networks. Each node corresponds to a random
variable $\text{The taint specification of method} M$, whose possible values are
$source$, $sink$, $sanitizer$, and $none$. Now, they are waiting for an external
oracle to instantiate its value to one of these four.

\subsection{Information Propagation}

Here, we expound on the core inference algorithm. Inference on the Bayesian
network is divided into two parts: local propagation and global propagation.
Local propagations pass information given by the oracle within a single graph,
whereas global propagations pass the information from one graph or several to
another. The Python library \emph{Pomegranate} is used to convert graphs into
Bayesian networks and perform local propagations.   % TODO: cite pomegranate

\paragraph{Local Propagations}

\subparagraph{Picking the Next Node}

We pick the next node based on an decision procedure called \emph{D-separation}.
This algorithm, given the following:

\begin{itemize}
\item A Bayesian Network (hence, a DAG),
\item A set of nodes $x$,
\item Another set of nodes $y$,
\item Yet another set of nodes $z$,
\end{itemize}

tells if nodes of $x$ may be probabilistically independent of those in $y$. We
utilize this algorithm like the following: For all nodes in a Bayesian network in
hand, we set $x$ as a singleton set of node $n_1$, $y$ as a singleton set of
node $n_2$, and $z$ as the set of all nodes that we have asked of its value so
far. This way, we can count the number of nodes that are sure to be
probabilisitically dependent to that node.

% 이 알고리즘은 그냥 말로 설명하려고 했는데 figure로 따로 빼야겠다.

% nx.d_separated(graph_for_reference, {node}, {other_node}, set(current_asked))

We use \emph{networkx}'s implementation of D-separation.

\subparagraph{Overview of the Local Propagation Algorithm}

Local propagations rely on loopy belief propagation algorithm (loopy BP): it is an
approximate version of an exact belief propagation algorithm (sum-product
algorithm), defined on Bayesian networks. Loopy BP may not converge,
nevertheless it provides a nice approximation to the original BP algorithm.
% TODO: cite https://www.jmlr.org/papers/volume6/ihler05a/ihler05a.pdf
Since the algorithm may not converge, we set the maximum number of iterations to
100, the default value set by \emph{Pomegranate}.

% TODO: should we explain a little on loopy BP?

% where should we insert the details of using d-separation?

Figure \emph{xx??} shows our local propagation algorithm. D-separation is used
when we pick the next node to ask of its specification. 

% insert algorithm here

% \label{sec:algorithm}
% \begin{algorithm}[t]
%   \caption{}
%   \begin{algorithmic}[1]
%     \Require{Bayesian Network $BN$ and transferred evidence $Ev$}
%     \Ensure{Mapping from all APIs to their labels}
%   \end{algorithmic}
% \end{algorithm}

\paragraph{Global Propagations}

Global propagations rely on a process which we call ``knowledge transfer''. It
relies on two major relations:

\begin{itemize}
\item similarity relation,
\item 1-call relation.
\end{itemize}

% insert algorithm here.

\section{Evaluation}

\subsection{RQs}
\begin{itemize}
\item How effective is SH in finding src/sin/san?  % I am really really worried about this
\item How efficient is SH?
\item Interaction user-study % possible?
\end{itemize}

% why is this meaningful? (with/without comparison)
% at least mention comparison

\subsection{Quantitative}

\subsection{Qualitative}

\paragraph{Discussions}
How well are propagations working? (Quant, Qualit) .
How stable is SpecHunter over multiple iterations?
\paragraph{Limitations}


\paragraph{Threats to Validity}



\section{Related Work}

% big TODO: Don't forget to mention Drake (PLDI'19)!!

% Conclusion itself must be 
\section{Conclusion}


% TODO cite pomegranate in the BibTeX


%% Acknowledgments
\begin{acks}                            %% acks environment is optional
  %% contents suppressed with 'anonymous'
  %% Commands \grantsponsor{<sponsorID>}{<name>}{<url>} and
  %% \grantnum[<url>]{<sponsorID>}{<number>} should be used to
  %% acknowledge financial support and will be used by metadata
  %% extraction tools.
  This material is based upon work supported by the
  \grantsponsor{GS100000001}{National Science
    Foundation}{http://dx.doi.org/10.13039/100000001} under Grant
  No.~\grantnum{GS100000001}{nnnnnnn} and Grant
  No.~\grantnum{GS100000001}{mmmmmmm}. Any opinions, findings, and
  conclusions or recommendations expressed in this material are those
  of the author and do not necessarily reflect the views of the
  National Science Foundation.
\end{acks}


%% Bibliography
% \bibliography{bibfile}


%% Appendix
\appendix
\section{Appendix}

Text of appendix \ldots

\end{document}
